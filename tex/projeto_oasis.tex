% Options for packages loaded elsewhere
% Options for packages loaded elsewhere
\PassOptionsToPackage{unicode}{hyperref}
\PassOptionsToPackage{hyphens}{url}
\PassOptionsToPackage{dvipsnames,svgnames,x11names}{xcolor}
%
\documentclass[
  oneside,
  open=any]{article}
\usepackage{xcolor}
\usepackage{amsmath,amssymb}
\setcounter{secnumdepth}{-\maxdimen} % remove section numbering
\usepackage{iftex}
\ifPDFTeX
  \usepackage[T1]{fontenc}
  \usepackage[utf8]{inputenc}
  \usepackage{textcomp} % provide euro and other symbols
\else % if luatex or xetex
  \usepackage{unicode-math} % this also loads fontspec
  \defaultfontfeatures{Scale=MatchLowercase}
  \defaultfontfeatures[\rmfamily]{Ligatures=TeX,Scale=1}
\fi
\usepackage{lmodern}
\ifPDFTeX\else
  % xetex/luatex font selection
\fi
% Use upquote if available, for straight quotes in verbatim environments
\IfFileExists{upquote.sty}{\usepackage{upquote}}{}
\IfFileExists{microtype.sty}{% use microtype if available
  \usepackage[]{microtype}
  \UseMicrotypeSet[protrusion]{basicmath} % disable protrusion for tt fonts
}{}
\makeatletter
\@ifundefined{KOMAClassName}{% if non-KOMA class
  \IfFileExists{parskip.sty}{%
    \usepackage{parskip}
  }{% else
    \setlength{\parindent}{0pt}
    \setlength{\parskip}{6pt plus 2pt minus 1pt}}
}{% if KOMA class
  \KOMAoptions{parskip=half}}
\makeatother
% Make \paragraph and \subparagraph free-standing
\makeatletter
\ifx\paragraph\undefined\else
  \let\oldparagraph\paragraph
  \renewcommand{\paragraph}{
    \@ifstar
      \xxxParagraphStar
      \xxxParagraphNoStar
  }
  \newcommand{\xxxParagraphStar}[1]{\oldparagraph*{#1}\mbox{}}
  \newcommand{\xxxParagraphNoStar}[1]{\oldparagraph{#1}\mbox{}}
\fi
\ifx\subparagraph\undefined\else
  \let\oldsubparagraph\subparagraph
  \renewcommand{\subparagraph}{
    \@ifstar
      \xxxSubParagraphStar
      \xxxSubParagraphNoStar
  }
  \newcommand{\xxxSubParagraphStar}[1]{\oldsubparagraph*{#1}\mbox{}}
  \newcommand{\xxxSubParagraphNoStar}[1]{\oldsubparagraph{#1}\mbox{}}
\fi
\makeatother


\providecommand{\tightlist}{%
  \setlength{\itemsep}{0pt}\setlength{\parskip}{0pt}}\usepackage{longtable,booktabs,array}
\usepackage{calc} % for calculating minipage widths
% Correct order of tables after \paragraph or \subparagraph
\usepackage{etoolbox}
\makeatletter
\patchcmd\longtable{\par}{\if@noskipsec\mbox{}\fi\par}{}{}
\makeatother
% Allow footnotes in longtable head/foot
\IfFileExists{footnotehyper.sty}{\usepackage{footnotehyper}}{\usepackage{footnote}}
\makesavenoteenv{longtable}
\usepackage{graphicx}
\makeatletter
\newsavebox\pandoc@box
\newcommand*\pandocbounded[1]{% scales image to fit in text height/width
  \sbox\pandoc@box{#1}%
  \Gscale@div\@tempa{\textheight}{\dimexpr\ht\pandoc@box+\dp\pandoc@box\relax}%
  \Gscale@div\@tempb{\linewidth}{\wd\pandoc@box}%
  \ifdim\@tempb\p@<\@tempa\p@\let\@tempa\@tempb\fi% select the smaller of both
  \ifdim\@tempa\p@<\p@\scalebox{\@tempa}{\usebox\pandoc@box}%
  \else\usebox{\pandoc@box}%
  \fi%
}
% Set default figure placement to htbp
\def\fps@figure{htbp}
\makeatother

\makeatletter
\@ifpackageloaded{caption}{}{\usepackage{caption}}
\AtBeginDocument{%
\ifdefined\contentsname
  \renewcommand*\contentsname{Table of contents}
\else
  \newcommand\contentsname{Table of contents}
\fi
\ifdefined\listfigurename
  \renewcommand*\listfigurename{List of Figures}
\else
  \newcommand\listfigurename{List of Figures}
\fi
\ifdefined\listtablename
  \renewcommand*\listtablename{List of Tables}
\else
  \newcommand\listtablename{List of Tables}
\fi
\ifdefined\figurename
  \renewcommand*\figurename{Figure}
\else
  \newcommand\figurename{Figure}
\fi
\ifdefined\tablename
  \renewcommand*\tablename{Table}
\else
  \newcommand\tablename{Table}
\fi
}
\@ifpackageloaded{float}{}{\usepackage{float}}
\floatstyle{ruled}
\@ifundefined{c@chapter}{\newfloat{codelisting}{h}{lop}}{\newfloat{codelisting}{h}{lop}[chapter]}
\floatname{codelisting}{Listing}
\newcommand*\listoflistings{\listof{codelisting}{List of Listings}}
\makeatother
\makeatletter
\makeatother
\makeatletter
\@ifpackageloaded{caption}{}{\usepackage{caption}}
\@ifpackageloaded{subcaption}{}{\usepackage{subcaption}}
\makeatother

\usepackage{hyphenat}
\usepackage{ifthen}
\usepackage{calc}
\usepackage{calculator}



\usepackage{graphicx}
\usepackage{geometry}
\usepackage{afterpage}
\usepackage{tikz}
\usetikzlibrary{calc}
\usetikzlibrary{fadings}
\usepackage[pagecolor=none]{pagecolor}


% Set the titlepage font families







% Set the coverpage font families

\usepackage{bookmark}
\IfFileExists{xurl.sty}{\usepackage{xurl}}{} % add URL line breaks if available
\urlstyle{same}
\hypersetup{
  pdftitle={Projeto},
  pdfauthor={Evandro Farias RochaAdjunto},
  colorlinks=true,
  linkcolor={blue},
  filecolor={Maroon},
  citecolor={Blue},
  urlcolor={Blue},
  pdfcreator={LaTeX via pandoc}}


\title{Projeto}
\usepackage{etoolbox}
\makeatletter
\providecommand{\subtitle}[1]{% add subtitle to \maketitle
  \apptocmd{\@title}{\par {\large #1 \par}}{}{}
}
\makeatother
\subtitle{Oásis de Cabaceiras}
\author{Evandro Farias Rocha\nCoordenador Adjunto}
\date{}
\begin{document}
%%%%% begin titlepage extension code


\begin{titlepage}

%%% TITLE PAGE START

% Set up alignment commands
%Page
\newcommand{\titlepagepagealign}{
\ifthenelse{\equal{center}{right}}{\raggedleft}{}
\ifthenelse{\equal{center}{center}}{\centering}{}
\ifthenelse{\equal{center}{left}}{\raggedright}{}
}


\newcommand{\titleandsubtitle}{
% Title and subtitle
{{\huge{\bfseries{\nohyphens{Projeto}}}}\par
}%

\vspace{\betweentitlesubtitle}
{
{\Large{\nohyphens{Oásis de Cabaceiras}}}\par
}}
\newcommand{\titlepagetitleblock}{
\newcommand{\HRule}{\rule{\linewidth}{0.5mm}} 

\HRule\\[0.4cm]

\titleandsubtitle

\HRule\\
}
\newcommand{\authorstyle}[1]{{\textsc{#1}}}

\newcommand{\affiliationstyle}[1]{{\large{#1}}}

\newcommand{\titlepageauthorblock}{
{\authorstyle{\nohyphens{Evandro Farias Rocha\nCoordenador Adjunto}\\}}
}

\newcommand{\titlepageaffiliationblock}{}
\newcommand{\headerstyled}{%
{\textsc{\LARGE{}}}
}
\newcommand{\footerstyled}{%
{}
}
\newcommand{\datestyled}{%
{\large{}}
}


\newcommand{\titlepageheaderblock}{\headerstyled}

\newcommand{\titlepagefooterblock}{
\footerstyled
}

\newcommand{\titlepagedateblock}{
\datestyled
}

%set up blocks so user can specify order
\newcommand{\titleblock}{\newlength{\betweentitlesubtitle}
\setlength{\betweentitlesubtitle}{\baselineskip}
{

{\titlepagetitleblock}
}

\vspace{1.5cm}
}

\newcommand{\authorblock}{{\titlepageauthorblock}

\vspace{0pt}
}

\newcommand{\affiliationblock}{{\titlepageaffiliationblock}

\vspace{0pt}
}

\newcommand{\logoblock}{{\includegraphics[width=0.2\paperwidth]{quarto/img/img\_fcf\_v3.png}}

\vspace{10\baselineskip}
}

\newcommand{\footerblock}{}

\newcommand{\dateblock}{}

\newcommand{\headerblock}{}

\thispagestyle{empty} % no page numbers on titlepages


\newlength{\minipagewidth}
\setlength{\minipagewidth}{\textwidth}
\raggedright % single minipage
% [position of box][box height][inner position]{width}
% [s] means stretch out vertically; assuming there is a vfill
\begin{minipage}[b][\textheight][s]{\minipagewidth}
\titlepagepagealign
\headerblock

\logoblock

\titleblock

\authorblock

\vfill

\dateblock
\par

\end{minipage}\ifthenelse{\equal{}{right} \OR \equal{}{leftright} }{
\hspace{\B}
\vrulecode}{}
\clearpage
%%% TITLE PAGE END
\end{titlepage}
\setcounter{page}{1}

%%%%% end titlepage extension code

\subsection{Introdução}\label{introduuxe7uxe3o}

\hfill\break
A Pesquisa da Pecuária Municipal (PPM) do IBGE tem como objetivo
fornecer informações estátisticas sobre o efetivo dos rebanhos
(cabeças), produtos de oriegem animal e produção agrícola. O inquérito é
anual e atinge todo o território nacional, com informações para o
Brasil, regiões geográficas, unidades federativas, mesorregiões e
microrregiões geográficas e municípios (IBGE, 2021).

Neste documento são apresentadas informações extraídas da PPM 2021 para
os rebanhos de caprinos e ovinos do estado da Paraíba. Para efeito de
análise comparativa foram considerados os dados de 2010 a 2021, além de
representação gráfica com dados de 1974 a 2021.

A fonte dos dados é a tabela 3939 do Sistema IBGE de Recuperação
Automática (SIDRA). Para coleta, tratamento e visualização dos dados
foram utilizados ferramentas de APIs (\emph{Application Programming
Interface}) que integram as arquiteturas das linguagens Python e R.\\

\subsection{1. Evolução dos Rebanhos no Brasil, Nordeste e
Paraíba}\label{evoluuxe7uxe3o-dos-rebanhos-no-brasil-nordeste-e-parauxedba}

\hfill\break
Considerando o período entre 2010 e 2021, o efetivo de caprinos no
Brasil passou de 9,31 para 11,9 milhões de cabeças, um aumento de 28\%.
No mesmo período, o rebanho de ovinos registrou um aumento de 18\%
passando 17,3 para 30.5 milhões de cabeças. O Nordeste apresentou melhor
desempenho, com taxas de crescimento de 34\% (caprinos) e 45\% (ovinos).
No caso da Paraíba, vale destacar o desempenho da ovinocultura com um
aumento de 71\%, passando de 433 para 744.1 mil cabeças. A
caprinocultura não apresentou o mesmo dinamismo, a taxa de crescimento
no período foi de 27\%, um aumento no efetivo de 600.6 para 764.7 mil
cabeças, um desempenho inferior aos registrado no Brasil e Nordeste
(Tabela 1).

O Nordeste com 11.3 milhões de cabeças de caprinos, 95,2\% do rebanho
nacional em 2021 é, seguramente, a região de maior densidade de rebanho
de caprino do país. Esta alta concentração tem raízes na capacidade de
adaptação desses animais às condições ambientais do Semiárido
nordestino. Dos dez maiores Estados produtores de caprinos do Brasil,
sete estão localizados na região Nordeste (Magalhães et all, 2020).

A caprinocultura paraíbana alcançou, em 2021, um efetivo de 764,7 mil
cabeças, representando 6,74\% do efetivo nordestino. Em 2010 esses
números representavam, respectivamente, 600,7 e 7,1\%. Os dados revelam
que, mesmo com com o aumento do efetivo de 27\%, a Paraíba diminuiu sua
participação no rebanho de caprino do Nordeste. No que concerne ao
efetivo de ovinos, a Paraíba, em 2021, contabilizou um total de 744,1
mil cabeças, 5,18\% do efetivo nordestino. Em 2010 esses números
representavam, respectivamente, 433 mil e 4,39\%, demonstrando um
crescimento positivo, inclusive na participação regional.

\newpage

\textsc{\textbf{Tabela 1. Evolução dos rebanhos caprinos e ovinos no
Brasil, Nordeste e Paraíba}}

\begin{longtable}[]{@{}lccc@{}}
\toprule\noalign{}
Categoria & Efetivo 2010 & Efetivo 2021 & Tx\_Variação (\%) \\
\midrule\noalign{}
\endhead
\bottomrule\noalign{}
\endlastfoot
\textbf{Brasil} & & & \\
Caprino & 9.312784 & 11.923630 & 28 \\
Ovino & 17.380581 & 20.537474 & 18 \\
\textbf{Nordeste} & & & \\
Caprino & 8.458578 & 11.353363 & 34 \\
Ovino & 9.857754 & 14.359997 & 45 \\
\textbf{Paraíba} & & & \\
Caprino & 600.607 & 764.758 & 27 \\
Ovino & 433.032 & 744.132 & 71 \\
\end{longtable}

\hfill\break
A Figura 1 apresenta a participação dos estados do nordeste no efetivo
do rebanho caprino da região Nordeste. O estado da Bahia ocupa a
primeira posição com um efetivo de 3.3 milhões de cabeças, 29,59\%,
seguido dos estados de Pernambuco, Piauí, Ceará e Paraíba,
respectivamente, com 3,2 28,22\%, 1,9, 17,14\%, 1,1, 10,2\% e 0,76,
6,74\% milhões de cabeças. Nas últimas posições se encontram os estados
do Rio Grande do Nordeste com 0,44, 3,95\%, Maranhão com 0,36, 3,17\%,
Alagoas com 0,08, 0,72\% e Sergipe com 0,024, 0,21\%.\\

Com relação ao efetivo de ovinos (Figura 2), a Bahia segue na liderança,
como um rebanho 4,2 milhões de cabeças, 29,58\% do efetivo nordestino.
Na sequência figuram os estados de Pernambuco com 3,4, 23,92\%, Ceará
2,5, 17,42\%, Piauí 1,7, 12,1\%, Rio Grande do Norte 0,87, 6,13\% e
Paraíba, ocupando a sexta posição com um efetivo de 0,74 milhões de
cabeças, 5,18\% do efetivo nordestino. Os três últimos estados que
compõe o bloco são Alagoas com 0,33, 2,35\%, Maranhão 0,29, 2,08\% e
Sergipe 0,17, 1,24\%.

\subsection{4. Nordeste: efetivo de caprinos e ovinos (Série
Histórica)}\label{nordeste-efetivo-de-caprinos-e-ovinos-suxe9rie-histuxf3rica}

\hfill\break
A partir da segunda metade da década de 1990, o Nordeste inicia uma
trajetória com taxas de crescimento ascendentes, principalmente para o
rebanho de ovino (Figura 3). Essa tendência, como pode ser observado na
Figura 4, fica mais evidente nos estados de Perrnambuco, Ceará, Paraíba
e Rio Grande do Norte. De forma geral os rebanhos de caprinos e ovinos
seguem a mesma tragetôria, o que leva a crer que ambos são afetados
pelas mesmas variáveis que influenciam no aumento ou diminuição do
número de animais (Monteiro et all, 2021).

Fatores climáticos e políticas públicas impactam diretamente na evalução
dos rebanhos. No caso das políticas públicas, por exemplo, o Programa de
Aquisição de Alimentos (PPA), em sua modalidade PPA-Leite, que
estabelecia parceria entre o governo federal e os esdados do Semiárido,
desempenhou papel central na manutenção da taxa de crescimento dos
rebanhos, impactando diretamente na capacidade de alocação de recursos
financeiros para a produção do leite caprino, participação e integração
de novos agentes ao setor produtuvo, dinamização da produção e estímulo
ao consumo. Essa política, inserida no âmbito do programa institucional
de combate à probreza e à fome, buscava garantir o fomento à produção e
ao consumo, e garantia de alimentos para população mais vunerável
beneficiária de programas como o Fome Zero.

\subsection{5. Paraíba: efetivo de caprinos e ovinos (Série
Histórica)}\label{parauxedba-efetivo-de-caprinos-e-ovinos-suxe9rie-histuxf3rica}

\hfill\break
Na Paraíba, considerando a série analisada, Figura 4, verifica-se um
crescimento do rebanho caprino a partir da segunda metade da década de
1990, alcançando um efetivo de 657,8 mil cabeças em 2005. Comparando o
efetivo de 2005 com 2012, constata-se uma redução do efetivo de 28\%,
184,7 mil cabeças a menos. Já com relação ao efetivo de ovinos, a série
histórica apresenta estabilidade, com algumas variações sazonais até o
ano de 2012.




\end{document}
